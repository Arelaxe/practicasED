\hypertarget{repConjunto_Letras_invConjunto_Letras}{}\section{Invariante de la representación}\label{repConjunto_Letras_invConjunto_Letras}

\begin{DoxyItemize}
\item No puede haber letras repetidas
\item No puede haber dos o más letras que tengan el mismo caracter pero distinto número de apariciones posibles y/o puntuación
\item Todas las letras del conjunto deben ser letras válidas (véase T\+DA \hyperlink{classLetra}{Letra}) 
\end{DoxyItemize}\hypertarget{repConjunto_Letras_faConjunto_Letras}{}\section{Función de abstracción}\label{repConjunto_Letras_faConjunto_Letras}
Un objeto válido {\itshape rep} del T\+DA \hyperlink{classConjunto__Letras}{Conjunto\+\_\+\+Letras} debe seguir la siguiente representación\+:
\begin{DoxyItemize}
\item Está formado por un set (S\+TL) de \hyperlink{classLetra}{Letra}, que permite que las letras estén ordenadas alfabéticamente y no puedan estar repetidas. De esta forma se pueden consultar sus datos muy fácilmente 
\end{DoxyItemize}